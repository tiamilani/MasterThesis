\chapter*{Sommario} % senza numerazione
\label{sommario}

\addcontentsline{toc}{chapter}{Sommario} % da aggiungere comunque all'indice

Lorem ipsum dolor sit amet, consectetur adipiscing elit. Donec sed nunc orci. Aliquam nec nisl vitae sapien pulvinar dictum quis non urna. Suspendisse at dui a erat aliquam vestibulum. Quisque ultrices pellentesque pellentesque. Pellentesque egestas quam sed blandit tempus. Sed congue nec risus posuere euismod. Maecenas ut lacus id mauris sagittis egestas a eu dui. Class aptent taciti sociosqu ad litora torquent per conubia nostra, per inceptos himenaeos. Pellentesque at ultrices tellus. Ut eu purus eget sem iaculis ultricies sed non lorem. Curabitur gravida dui eget ex vestibulum venenatis. Phasellus gravida tellus velit, non eleifend justo lobortis eget.


  Sommario è un breve riassunto del lavoro svolto dove si descrive l'obiettivo, l'oggetto della tesi, le 
metodologie e le tecniche usate, i dati elaborati e la spiegazione delle conclusioni alle quali siete arrivati.  

Il sommario dell’elaborato consiste al massimo di 3 pagine e deve contenere le seguenti informazioni:
\begin{itemize}
  \item contesto e motivazioni 
  \item breve riassunto del problema affrontato
  \item tecniche utilizzate e/o sviluppate
  \item risultati raggiunti, sottolineando il contributo personale del laureando/a
\end{itemize}




