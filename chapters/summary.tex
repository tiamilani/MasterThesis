\chapter*{Summary} % senza numerazione
\label{cha:summary}

\addcontentsline{toc}{chapter}{Summary} % da aggiungere comunque all'indice

%\begin{itemize}
%		\item There are few studies about the problem
%		\item FSM explosion because of the path exploration problem
%		\item MRAI interaction with the problem
%		\item RFD interaction with the problem
%		\item Interaction between the mechanisms
%\end{itemize}

\rlc{Il sommario può anche essere più lungo di 1 pagina, io aggiungerei un paragrafo sui ``goal'' della tesi prima di ``For this reason \ldots'' \\ 
Rivedi ancora l'inglese, la punteggiatura è approssimativa, non si inizia mai una frase con That o But (meglio However, ma ci sono anche altre forme)}

%Presentation of BGP and the problems
\ac{BGP} is the protocol daily used on the Internet to propagate changes in the
reachability of the publicly available networks.
Like all the other network protocols, \ac{BGP} uses control messages to distribute
information to all the other nodes of the network.
The tune of multiple parameters can influence the general performances of the
network.
But, it has been proven in~\cite{fabrikant2011there} that an uncontrolled configuration
of those parameters could provoke an exponential explosion in the number of messages
and by consequence also increase the general convergence time.
This problem is called \textit{Path Exploration} and it arises when a node shares
a sequence of non-optimal choices before actually distribute the best possible
decision that eventually occurs.

\ac{BGP} is also a protocol that emphasizes this particular behaviour because of
its tendency to act as an echo-chamber for every new information.
A \ac{BGP} node shares its decision with all the neighbours that respect
its set of policies, provoking a cascade effect.
This is the first cause of \q{noise} in \ac{BGP}, the inherent noise, the second
source is external and could be represented by a \ac{BGP} node that frequently
changes its decisions because of a faulty interface or wrong configurations.

%Hypothesis
\ac{BGP} implements two different mechanisms to reduce the effects of the
noise that otherwise would make the protocol unusable.
The first one is \ac{MRAI} and recent studies on it show that the correct
configuration of it can lead to more efficient use of the
resources~\cite{griffin2001experimental,fabrikant2011there,deshpande2004impact,milani2020improving}.
The second type of noise is mitigated by \ac{BGP} using the \ac{RFD} mechanisms.
That uses the incoming messages in order to evaluate if an incoming route has
to be suppressed because too noisy.
Multiple studies on it show how it can be tuned to have better performances in
the case of small variations on the network, but still, be effective on heavily
flapping routes~\cite{mao2002route,gray2020bgp,rfc7196}.

Those two mechanisms use the same triggering event to take any action, the reception
of a message with new information.
The main hypothesis behind this thesis is that there is actually an interaction
between \ac{MRAI} and \ac{RFD}.
The goal of this thesis is to study the factors that influence \ac{MRAI} and
\ac{RFD} separately and then verify that an interaction actually exists between
them.

%Studies in the thesis
For this reason, I developed a software tool chain that permits to simulate and
study the evolution of \ac{BGP} synthetic graphs using a \ac{DES}.
Thanks to the high adaptability of the environment has been possible to tune
the parameters with different strategies and compare them.
Not all the nodes react in the same way to different changes, thanks to this
platform it has been also possible to study the evolution of single nodes
with respect to the average performances of the network.

Thanks to the results of all the simulations it was possible to confirm
the intuition at the basis of this thesis.
Multiple factors can influence the performances of \ac{MRAI} and \ac{RFD} with
various different outcomes in the evolution of the network.
Also, I have been able to confirm the hypothesis presented in~\cite{griffinFSM,fabrikant2011there}
on the effects of the \textit{Path Exploration} problem are correct.

%My contribution to the thesis
My contribution to this thesis can be summed up in the following points:
\begin{itemize}
		\item Deployment of a software tool chain that permits to experiment
		with \ac{BGP} networks and multiple parameters combined in order to study
		different possible evolutions;
		\item Verification of the theoretical hypothesis presented in
		\cite{griffinFSM,fabrikant2011there} through a \ac{FSM} generated by
		multiple simulations;
		\item Study on the factors that can influence \ac{MRAI} and \ac{RFD} and
		by consequence the network performances.
		\item Verification that an interaction actually exists between \ac{MRAI}
		and \ac{RFD} and that \ac{MRAI} can have a higher influence.
\end{itemize}
