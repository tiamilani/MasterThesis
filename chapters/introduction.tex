\chapter{Introduction}
\label{cha:introduction}

%\begin{itemize}
%    \item How is internet built
%    \item the protocol that controls internet
%\end{itemize}

%\fxfatal{Expand the concepts}

With the name \q{The Internet} we define a network composed of more than \num{60000}
entities that share their knowledge in order to permit us to reach every website,
identified by a unique prefix, whenever we want.
A prefix is the translation from the textual form to the \ac{IP} addresses
standard representation used on the Internet.
It is composed by two parts, the address and a suffix that permits to identify the
prefix set, respecting the \ac{CIDR} notation \cite{fuller1993classless}.
We are used to thinking about the Internet as something far away from us, something
that we do not have to care about, leaving all the complexity
out.
However, from a more physical point of view, what is the Internet? It is nothing
more than a big network where interconnected entities keep the prefixes reachable.
Those entities are in reality called \acp{AS} and their function is to hold and
control some \ac{IP} prefixes used by one or more operators.

Every \ac{AS} is responsible for the connectivity to the prefixes that it shares.
We know that networks are able to react to changes thanks to routing protocols,
and the Internet does not differ on that.
Every \ac{AS} has to keep active its own instance of an Internet routing protocol
so that it can react to changes.
This routing protocol is the glue of the Internet, is what permits
us to always be able to reach the other side of the world without knowing
the actual route that our packets take.

The path that we use could change because of different factors that could
be technical, economical or even political.
This is because relationships between \acp{AS} are controlled by contracts and different
arrangements may have different fees applied to the transmission of the knowledge.
Some paths may be used only as backups if the primary one fails, or even used
only for certain types of traffic flows.
These policies must be implementable in the Internet routing protocol that has
to discern on which path, among all the known alternatives, is the best one
considering the \ac{AS} convenience.

The protocol that has been designed to handle these situations is  \ac{BGP}.
It has been released in \num{1989} and is in use on the Internet since \num{1994}.
It reached its last version, the fourth one, in \num{2006} \cite{rfc4271}.
Is easy to imagine that in almost \num{30} year \ac{BGP} is changed a lot from
the beginning and also the needs of the different \acp{AS} are changed a lot
because of the technology improvements.
Up to now, \ac{BGP} can be expanded with tens of \acp{RFC} that improve the
range of possibilities, those optional parameters are actually very important
for the \acp{AS} because the complexity of the relationships is growth a lot
in the last years.

\ac{BGP} is an instance of the \textit{Bellman-Ford} distance vector routing
protocol that shares, besides the prefixes known by the speakers, also the
path used to reach the destination.
Other than that, to control all the possible policies applied by the \acp{AS},
\ac{BGP} implements also different parameters and attributes that can be
personalized.
In \ac{BGP} multiple parameters play a central role, and the correct setting
of them could influence the performances of, not only the single \ac{AS} but,
the entire network.
For this reason, the research is still active to find new technologies and
a trade-off, between convergence time and messages transmitted, that could
be sustainable by the current hardware.

%\rlc{Non passare in prima persona qui e ricorda che in uno scritto formale non si usano le abbreviazioni tipo I'm we're e così via}
In this thesis, more precisely in \Cref{cha:bgp_art}, two of those important
mechanisms will be introduced, \ac{MRAI} and \ac{RFD}.
The first one is used to compress multiple input messages into one
output packet in order to reduce the network load provoked by a change.
The second one, \ac{RFD}, is used to penalize unstable paths, suppressing
the route and blocking the spreading of it to further nodes to circumscribe
the zone of instability.

The existence of those two mechanisms has been studied separately many different
times \cite{fabrikant2011there,daggitt2018rate,qiu2005optimal,gray2020bgp}.
Although, there are almost no studies on the interaction of them.
Even if the effects of one interact with the other.

\section{Internet Today}
\label{sec:internet_today}

Internet, as a network, is constantly growing, in terms of \acp{AS}, prefixes
and messages transmitted.
This continuous growth increase also the load on the \ac{BGP} nodes that
receives more messages and have to manage the effects in terms of memory
and processing power.
As a consequence, this increases also the load on the network, because of the
nature of \ac{BGP} to act as echo-chamber.

Thanks to the annual report from \ac{APNIC} we can have a snapshot of
the situation of the Internet and the evolution of it.
The data collected by \ac{APNIC} from \num{2007} concern the knowledge
of the \ac{AS} 131072 that has two links with other \acp{AS}, one in Japan and
the other one in Australia\footnote{\href{https://blog.apnic.net/2021/01/05/bgp-in-2020-the-bgp-table/}{source APNIC data}}.
Prefixes of smaller and smaller sizes are continuously shared, the number of
%\rlc{Non hai definito la notazione CIDR quindi $/24$  non ha senso se non per gli iniziati. Introducila prima quando spieghi cos'è un prefisso \ldots così poi puoi usare una terminologia più tecnica senza ambiguità}
networks with a $/24$ netmask distributed in the last year has been growing constantly,
fomenting the problem described above.

This redundancy of the \ac{BGP} nodes provokes the \textit{Path Exploration}
problem.
This particular issue occurs when a node enters a transitory state where it
continuously shares non-optimal paths while it doesn't reach a stable state.
Provoking the propagation of non-ideal routes to other nodes causing a vicious
circle.
The growing of the Internet is not negligible because of this problem, a continuous
growth in terms of nodes and edges cause the growth of favorable conditions for
the \textit{Path Exploration}.

\section{Interaction between variables and convergence}
\label{sec:bgp_correlations}

The two parameters  studied in this thesis are \ac{MRAI} and \ac{RFD}
and how the interaction between them works.
Our first hypothesis is that, indeed, there is an interaction.
This hypothesis is sustained by the fact that both mechanisms operate to
reduce the noise of \ac{BGP} (defined in \Cref{sec:bgp_noise}).
Both have different parameters and different behaviours,
but, if a node wants to transmit a message it must respect \ac{MRAI} and the input
could be caused by \ac{RFD} that suppress/reintroduce a route.
In the opposite case, a too small \ac{MRAI} value could permit different message
storms that would trigger the \ac{RFD} suppression systems.

One of the goals of reducing the value of \ac{MRAI} is to reach a faster convergence
paying the cost of more messages.
Unfortunately, looking only at \ac{MRAI} is not possible to get reliable results
for general purposes, in-fact, like showed in \Cref{cha:bgp_rfd_vs_mrai}, is possible
to obtain the opposite result due to the fact that to solve \ac{RFD} suppression
is required a longer time.

In the opposite case, if \ac{RFD} is configured in the wrong way is possible to
end up to be too much permissive, leaving \ac{MRAI} to handle all the noise and
it could be not effective if the storms are sufficiently delayed in time.

For those reasons is important to study these two parameters together, because
there could be a strong co-dependence.

\section{The goal of this thesis}
\label{sec:thesis_goal}

%\begin{itemize}
%    \item Why is important understand this correlation?
%\end{itemize}

The goal of this thesis is to prove that the noise reduction mechanisms of \ac{BGP}
interact with one another, studying this interaction through simulative experiments
and from those give useful hints on how the two parameters interact with one another.
In order to build the basis for future experiments that can study more deeply
and maybe in a formal way the phenomena.
Is also mandatory for this thesis to develop the platform where those experiments
would be executed and make that platform publicly
available\footnote{\href{https://github.com/tiamilani/BGPFSM}{GitHub repository}}.

The protocol and the two mechanisms are more deeply explained in \Cref{cha:bgp_art},
while, in \Cref{cha:des} the experimentation system is explained.
This system will be used in
\Cref{cha:bgp_fsm,cha:bgp_mrai_experiments,cha:bgp_rfd,cha:bgp_rfd_vs_mrai}
to perform the experiments about the \textit{Path Exploration} problem and then
how \ac{MRAI} and \ac{RFD} can impact the performances, conclusions to follow
in \Cref{cha:conclusion}.

The \ac{BGP} community has not yet reached a common agreement on what values
to use; for this reason I will evaluate different possible techniques that can
be applied to both the parameters and comment on the network performances obtained.

