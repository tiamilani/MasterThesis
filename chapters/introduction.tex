\chapter{Introduction}
\label{cha:introduction}

%\begin{itemize}
%    \item How is internet built
%    \item the protocol that controls internet
%\end{itemize}

%\fxfatal{Expand the concepts}

With the name \q{The Internet} we define a network composed by more than \num{60000}
entities that share their knowledge in order to permits us to reach every website
whenever we want.
We are used to think about Internet as something far away from us, something
that we do not have to care about, we can use it leaving all the complexity
out.
But, by a more physical point of view, what is the Internet? It is nothing
more than a big network where interconnected entities keeps the prefixes reachable.
Those entities are in reality called \acp{AS} and their function is to hold and
control some \ac{IP} prefixes used by one or more operators.

Every \ac{AS} is responsible for the connectivity to the prefixes that it shares.
We know that the networks are able to react to changes thanks to routing protocols,
and Internet is not different in that.
Every \ac{AS} has to keep active its own instance of an Internet routing protocol
in order react to changes.
This routing protocol is the glue of the Internet, its the thing that permits
us to always be able to reach the other side of the world without knowing
the actual route that our packets take.

The path that we use could change because of different factors that could
be technical, economical or even political.
Thats because the \ac{AS} relationships are controlled by contracts and different
contracts can have different fees applied for the transmission of the knowledge.
Some paths may be used only as backups if the primary one fail, or even used
only for certain type of traffic flows.
This policies must be implementable in the Internet routing protocol that has
to discern on which path, among all the known alternatives, is the best one
considering the \ac{AS} convenience.

The protocol that has been created to handle this situations is the \ac{BGP}.
It has been released in the \num{1989} and is in use on the Internet since
the \num{1994}.
It reached its last version, the fourth one, in \num{2006} \cite{rfc4271}.
Is easy to imagine that in almost \num{30} year \ac{BGP} is changed a lot from
the beginning and also the needs of the different \acp{AS} are changed a lot
because of the technology improvements.
Up to now, \ac{BGP} can be expanded with tens of \acp{RFC} that improve the
range of possibilities, those optional parameters are actually very important
for the \acp{AS} because the complexity of the relationships is growth a lot
in the last years.

\ac{BGP} is an instance of the \textit{Bellman-Ford} distance vector routing
protocol that shares besides the prefixes known by the speakers also the
path used to reach the destination.
Other than that, to control all the possible policies applied by the \acp{AS},
\ac{BGP} implements also different parameters and attributes that can be
personalized.
In \ac{BGP} multiple parameters play a central role, and the correct setting
of them could influence the performances of, not only the single \ac{AS} but
the entire network.
For this reason, the research is still active to find new technologies and
a trade-off between, convergence time and messages transmitted that could
be sustainable by the current hardware.

In this thesis, more precisely in \Cref{cha:bgp_art}, I'm going to introduce
two of those important \ac{BGP} mechanisms, \ac{MRAI} and \ac{RFD}.
The first one is used to compress multiple input messages into one
output message in order to reduce the network load provoked by a change.
The second one, \ac{RFD}, is used to penalize unstable paths, suppressing
the route and blocking the spreading of it to further nodes to circumscribe
the zone of instability.

The existence of those two mechanisms has been studied separately many different
times \cite{fabrikant2011there,daggitt2018rate,qiu2005optimal,gray2020bgp}.
But, there are almost no studies on the interaction of them.
Even if the effects of one interact with the other.

\section{Internet Today}
\label{sec:internet_today}

Internet, as a network, is constantly growing, in terms of \acp{AS}, prefixes
and messages transmitted.
This continuous growth increase also the load on the \ac{BGP} nodes that
receives more messages and have to manage the effects in terms of memory
and processing power.
As a consequence, this increases also the load on the network, because of the
nature of \ac{BGP} to act as echo-chamber.

Thanks to the annual report from \ac{APNIC} we can have a snapshot of
the situation of Internet and the evolution of it.
The data collected by \ac{APNIC} from \num{2007} concern the knowledge
of the \ac{AS} 131072 that has two links with other \acp{AS}, one in Japan and
the other one in Australia\footnote{\href{https://blog.apnic.net/2021/01/05/bgp-in-2020-the-bgp-table/}{source APNIC data}}.
Prefixes of smaller and smaller sizes are continuously shared, the number of
$/24$ networks distributed in the last year has been growing constantly.
Fomenting the problem described above, generating a vicious circle of messages.

This redundancy of the \ac{BGP} nodes provoke the \textit{Path Exploration}
problem.
This particular issue occurs when a node enter in a transitory state where it
continuously shares non optimal paths while it doesn't reach a stable state.
Provoking the propagation of non ideal routes to other nodes causing a vicious
circle.
The growing of Internet is not negligible because of this problem, a continuous
growth in terms of nodes and edges cause the growth of favorable conditions for
the \textit{Path Exploration} problem.

\section{Interaction between variables and convergence}
\label{sec:bgp_correlations}

The two parameters that will be studied in this thesis are \ac{MRAI} and \ac{RFD}
and how the interaction between them works.
Our first hypothesis is that, indeed, there is an interaction.
This hypothesis is sustained by the fact that both parameters operate to
reduce the noise of \ac{BGP}, with different parameters and different behaviour,
but, if a node wants to transmit a message it must respect \ac{MRAI} and the input
could be caused by \ac{RFD} that suppress/reintroduce a route.
On the opposite case, a too small \ac{MRAI} value could permit different message
storms that would trigger the \ac{RFD} suppression systems creating vicious circle.

One of the goals of reducing the value of \ac{MRAI} is to reach a faster convergence
paying the cost of more messages.
Unfortunately, looking only to \ac{MRAI} is not possible to get reliable results
for general purposes, in-fact, like showed in \Cref{cha:bgp_rfd_vs_mrai}, is possible
to obtain the opposite result due to the fact that to solve \ac{RFD} suppression
is required a longer time.

On the opposite case if we tune in the wrong way \ac{RFD} we would end up to
be too much permissive, leaving to handle all the noise to \ac{MRAI} that could
be not effective if the storms are sufficiently delayed in time.

For those reasons is important to study these two parameters together, because
there could be a strong co-dependence.

\section{The goal of this thesis}
\label{sec:thesis_goal}

%\begin{itemize}
%    \item Why is important understand this correlation?
%\end{itemize}

The goal of this thesis is to prove that the noise reduction mechanisms of \ac{BGP}
interact one another, studying this interaction through simulative experiments
and from those give useful hints on how those two parameters interact with one another.
In order to build the basis for future experiments that can study more deeply
and maybe in a formal way the phenomena.
Is also mandatory for this thesis to develop the platform where those experiments
would be executed and make that platform public
available\footnote{\href{https://github.com/tiamilani/BGPFSM}{GitHub repository}}.

In \Cref{cha:bgp_art} are going to be presented the protocol and more deeply
the two parameters studied, while in \Cref{cha:des} I'm going to present
the structure of the platform that will be used in
\Cref{cha:bgp_fsm,cha:bgp_mrai_experiments,cha:bgp_rfd,cha:bgp_rfd_vs_mrai}
to perform the experiments about the \textit{Path Exploration} problem and then
how \ac{MRAI} and \ac{RFD} can impact the performances, conclusions to follow
in \Cref{cha:conclusion}.

The \ac{BGP} community has not yet reached a common agreement on what values
to use for this reason I will evaluate different possible techniques that can
be applied to both the parameters and comment on the network performances obtained.

