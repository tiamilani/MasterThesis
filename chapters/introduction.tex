\chapter{Introduction}
\label{cha:introduction}

%\begin{itemize}
%    \item How is internet built
%    \item the protocol that controls internet
%\end{itemize}

\fxfatal{Expand the concepts}

What we call Internet is a single network that interconnects more than \num{60000}
\acp{AS} that share their knowledge in order to spread the subnets reachability.
An \ac{AS} is a single entity that holds and controls some \textit{IP} prefixes
used by one or more operators.
Every \ac{AS} is responsible for its own interconnectivity policies that controls
the spread of information obtained by the outside and the one from the inside.

\ac{BGP} is the only protocol used on the Internet that controlls the propagation 
of the knowledge between different \acp{AS}
\ac{BGP} has been released in \num{1989} and is in use on the internet since 
\num{1994}.
Since that year \ac{BGP} is growth a lot, it reaches its last version, number
fourth, in \num{2006} \cite{rfc4271}.
Up to now \ac{BGP} is composed by tens of \acp{RFC}, there are a lot of possibilities,
from policies to attributes used to tag messages and give more information than
just the path to reach the destination.

Multiple of its parameters play a central role in it, and the correct setting
of them could influence the performances of, not only the single \ac{AS} but
the entire network.
For this reason the research is still active to find new technologies and
trade-off between, convergence time and messages transmitted.

In this thesis, more precisely in \Cref{cha:bgp_art}, I'm going to introduce
two of them, \ac{MRAI}, used to compress multiple input messages into one 
output message and the \ac{RFD} parameter, that is used penalize unstable paths
suppressing routes.
Those two parameters goal is to control two different type of noise that 
\ac{BGP} creates/controls.

The existence of tose two parameter has been studied separatly many different
times, \cite{fabrikant2011there,daggitt2018rate,qiu2005optimal,gray2020bgp}.
But, there are almost no studies on the correlation of those parameters.
Even if the effects of one interact with the other.

\section{Internet nowadays}
\label{sec:internet_today}

Internet, as a network, is constantly growing, in terms of \acp{AS}, prefixes
and messages transmitted.
Increasing also the load on the \ac{BGP} nodes and by consequence the load
of the network in general.
There are always more smaller prefixes announced, the number of $/24$ prefixes
distributed in the last year has constantly growth.
Data obtained from the annual report of the \ac{AS} 131072 that register data
since \num{2007} from two APNIC \ac{BGP} speakers point, one located in Japan
and the other in Austrelia \footnote{\href{https://blog.apnic.net/2021/01/05/bgp-in-2020-the-bgp-table/}{source APNIC data}}.

This growth of the Internet is not negligible because of one, between the may, 
problem that it has due to the \ac{BGP} protocol.
\ac{BGP} suffers of the \textit{Path Exploration} problem, a node can enter in 
a transitory state where it continuosly shares non optimal paths while it
doesn't reach the stable state.
Even few tens of nodes can generate thousends of messages due to this problem 
\cite{deshpande2004impact}.

\section{Correlation between variables and convergence}
\label{sec:bgp_correlations}

The two parameters that will be studied in this thesis are \ac{MRAI} and \ac{RFD}
and how the interaction between them works.
Our first hypotesis is that, indeed, there is an interaction.
This hypotesis is substained by the fact that both parameters operates to 
control the noise of \ac{BGP}, with different parameters and different behaviour,
but, if a node want to transmit a message it must respect \ac{MRAI} and the input
could be caused by the \ac{RFD} that suppress/reintroduce a route. 
On the opposite case a to small \ac{MRAI} value could permit different message
storms that would trigger the \ac{RFD} suppression systems.

One of the goal of reducing the value of \ac{MRAI} is to reach a faster convergence
paying the cost of more messages.
Unfurtunatly, looking only to \ac{MRAI} is not possible to get reliable results
for general purposes, in-fact, like showed in \Cref{cha:bgp_rfd}, is possible 
to obtain the opposite result due to the fact that to solve \ac{RFD} suppressions
is required a longer time.

On the opposite case if we tune in the wrong way \ac{RFD} we would endup to
be too much permissive, leaving to handle all the noise to \ac{MRAI} that could
be not affective if the sorms are sufficently delayed in time.

For those reasons is important to study this two parameters togheter, because
there could be a strong co-dependance.

\section{Goal of this thesis}
\label{sec:thesis_goal}

%\begin{itemize}
%    \item Why is important understand this correlation?
%\end{itemize}

The goal of this thesis is to prove that the noise detectors of the \ac{BGP} 
protocol are correlated, studing this correlation through simulative experiments
and from that give useful hints on how those two parameters interact one another.
In order to build the basis for future experiments that can study more deeply 
and maybe in a more formal way the phenomen.
Is also mandotory for this thesis to develop the platform where those experiments
would be executed and make that platform publically available
\footnote{\href{https://github.com/tiamilani/BGPFSM}{GitHub repository}}.

In \Cref{cha:bgp_art} are going to be presented the protocol and more deeply
the two parameters studied, while in \Cref{cha:des} I'm going to present
the structure of the platform that will be used in \Cref{cha:bgp_fsm,cha:bgp_mrai_experiments,cha:bgp_rfd} 
to perform the experiments about the \textit{Path Exploration} problem and then
how \ac{MRAI} and \ac{RFD} can impact th performances, conclusions to follow
in \Cref{cha:conclusion}.

The \ac{BGP} community has not yet reached a common agreement on what values 
to use for this reason I will evaluate different possible techniques that can 
be applied to both the parameters and comment the network performances obtained.
