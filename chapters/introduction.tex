\chapter{Introduction}
\label{cha:introduction}

%\begin{itemize}
%    \item How is internet built
%    \item the protocol that controls internet
%\end{itemize}

\fxfatal{Expand the concepts}

What we call Internet is a single network that interconnects more than \num{60000}
\ac{AS} that shares their knowledge in order to spread the subnets reachability.
An \ac{AS} is a single entity that holds and controls some \textit{IP} prefixes
used by one or more operators.
Every \ac{AS} is responsible for its own interconnectivity policies that controls
the information obtained by the outside and the one from the inside.

The propagation of the knowledge between different \ac{AS} is controlled by 
\ac{BGP}. 
Is possible to configure the policies inside \ac{BGP} in order to apply different
choices to respect the constraints imposed by contract between the \ac{AS}es.

\ac{BGP} is growth a lot from its first release, it now include multiple optional
parameters that are actually more mandatory than optional.
It is the only protocol activelly used on the Internet for the spreading 
of \textit{IPv4} and \textit{IPv6} networks.

Multiple of its parameters play a central role in it, and the correct setting
of them could influence the performances of, not only the single \ac{AS} but
the entire network.
For this reason the research is still active to find new technologies and
trade-off between, for example, convergence time and messages transmitted.

But there are almost no studies on the correlation of those parameters.
Even if the effects of some of them trigger other parameters.

%Those commercial contracts will limits the \ac{AS} possibilities, for example
%is possible that a constraints could be that an \ac{AS} wouldn't pass through
%an other precise \ac{AS}

\section{Internet nowadays}
\label{sec:internet_today}

\begin{itemize}
    \item Use today studies to show how internet is today
\end{itemize}

\fxfatal{Maybe use some apnic data}

\section{Correlation between variables and convergence}
\label{sec:bgp_correlations}

%\begin{itemize}
%    \item Expose the hypothesis of the correlation
%\end{itemize}

\ac{BGP} is an intrisically noisy protocol, those noises could be due to different
factors.
Could be \ac{BGP} itself to produce noisy situations
There are some parameters of \ac{BGP} that point to intercept and limit noisy 
situations.
The first one is \ac{MRAI}, which is a timer with the goal to compress multiple
messages, in order that a single \ac{AS} can produce less messages.

Forget about the possibility to converge in seconds or even sub-seconds
when we talk about internet routing convergence there are a lot of factors
that influence it.
The convergence time is mostly affected by some timers that rules the Internet.
It could require up to different minutes to achieve a complete convergence, 
spread a new routing information to all the nodes.

One of the most effective timers is \ac{MRAI} and it has been already
proven \fxfatal{Insert citation} that whith 

\section{Goal of this thesis}
\label{sec:thesis_goal}

%\begin{itemize}
%    \item Why is important understand this correlation?
%\end{itemize}

Is important to study the correlation between those parameters because Internet
is not built on just one of them.
Tuning one parameter is important to understand the effects on the others, and
viceversa.

This thesis has the goal to gives a useful help to the understanding of this
protocol interoperation on a more global scale.
