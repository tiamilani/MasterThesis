\chapter{Conclusion}
\label{cha:conclusion}

In this thesis, I exposed different noise problems that \ac{BGP} contains and
studied the parameters that are used to curb the problem.
I have then analyzed the results from thousands of experiments in order to
provide a solid baseline that shows how \ac{MRAI} and \ac{RFD} are related
one another.
I have also shown that, due to the \textit{Path exploration} problem, even
in small networks is really difficult to infer the causes behind a transmitted
signal.

The instruments developed during this thesis are publicly available in the
hope that other scientist could use them to study properties of \ac{BGP} that
is an extremely vast protocol.

I have then analyzed how new techniques differ from the standard or legacy one,
like in the case of \ac{RFD} where the legacy values are still present on the
internet and can have a huge impact on the convergence.
I also studied the impact that can have, in terms of performances, \ac{MRAI} on \ac{RFD},
how after a certain threshold the gain obtained from the lower number of
suppression is ininfluent in terms of convergence time and messages transmitted.

This thesis creates the basis for studies on the interaction of \ac{BGP} parameters
and would also be a warning for those studies that point to improve only
one aspect of \ac{BGP}.
Remember to always look from a different perspective because what looks like an
improvement, on the one hand, could bring the overall performance of the
network to collapse.

%\begin{itemize}
%    \item Wrap up
%    \item Path exploration explosion of the FSM
%    \item MRAI convergence dependency
%    \item RFD and MRAI co-dependency
%\end{itemize}

\section{Future Works}
\label{sec:future_works}

The development of the platform to increment the number of features is one of
the major points in the future works, but during the experiments we have also
made some assumptions/restrictions to the environment.
Those restrictions could be relaxed to study more heterogeneous environments.

\subsubsection{Policies}

One of the first assumption is that every node accept everything comes from its
neighbours and redistribute it.
This is not always true, \acp{AS} on the internet can have any sort of policy,
checking any possible attribute of the message received.
A possible future work could be to study the bibliography behind those policies
in order to be able to implement them and study again the performances of
the network with those restrictions.

\subsubsection{Multiple destinations and path aggregation}

During the experiments I never introduced more than one destination subjected to a
signal, even if the \ac{DES} permits to have multiple of them.
Obviously more destinations could produce more messages and more \ac{ADV}
storms, but a possible interesting point could be to see the reactions of the nodes
with the path aggregation activated and how it can impact the performances.
