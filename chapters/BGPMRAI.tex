\chapter{BGP MRAI dependency}
\label{cha:bgp_mrai_experiments}

\ac{MRAI} is one of the parameters that mostly has caused divergences in the 
scientific community.
And, after the introduction in the protocol since the version 4 \cite{rfc4271}
\fxfatal{Check this sentence}, is one of the more studied for the possibility
to improve the protocol or generate exponential convergence behaviour in small
network \cite{fabrikant2011there}.

The protocol strictly depends on this parameter, because as we saw in \Cref{cha:bgp_fsm},
the incorrect use of it can lead to tremendous consequences, even worst of not
having it at all.
In other cases, with a particular setting of it is possible to improve the network
performances.
Recent studies about centrality metrics on routing protocols introduce, through 
the distributed computation of the metric, to a 
timer trade off improvement \cite{MaLo18_ToN,GhiMa18_infocom}.
This kind of approach has been also applied on \ac{BGP} with positive results on
network failures \cite{milani2019BGP,milani2020improving}.

All those study points out how we can set \ac{MRAI} to improve network
performances, but what about how \ac{MRAI} reacts on different problems?
Is it possible that \ac{MRAI} reacts differently based on where the signal
occurs?
In fact, our hypothesis is that is not enough just look to the \ac{MRAI} setting
because also other factors can be relevant.
For example, a change near the central clique of $T$ nodes could provoke a large
storm of messages because \ac{MRAI} doesn't affect in time the spreading of information.
While, a change in the periphery could be cushioned without it reaching the center
of the network.

\section{Clique graph}
\label{sec:bgp_mrai_clique}

The clique topology is one of the worst case scenario as specified in Labovitz et al.
\cite{labovitz2000delayed}
I used two approaches in this Environment, the first one keeps the \ac{IW} active
the second one avoid the use of this property.
To emphasize the effects of this parameter with the effects also of different 
\ac{MRAI} settings.

The Environment properties are listed in \Cref{tbl:clique_properties}

\begin{table}[h]
	\begin{center}
	\begin{tabular}{ || m{4cm}| m{8cm} || } 
	\hline
	Property & Value \\ 
	\hline \hline
	Seeds & $[1, 10]$ \\ 
	\hline
	Signaling & \q{AW} \\
	\hline
		Withdraws delay & Uniform distribution between \SI{1}{\second} and \SI{5}{\second} \\ 
	\hline
	Announcement delay & constant distribution of \SI{5}{\second} \\ 
	\hline
		MRAI & $[0, 60]$ \\
	\hline
	Link delay & Uniform distribution between \SI{0.0001}{\second} and \SI{0.5}{\second} \\
	\hline
	\end{tabular}
\end{center}

	\caption{Clique environment properties}
	\label{tbl:clique_properties}
\end{table}

As described in \Cref{tbl:clique_properties}, for each \ac{MRAI} value has been
executed \num{10} different runs of the environment.
The clique graph used in this experiments is composed by \num{15} nodes.
The \ac{MRAI} strategy used is the \textit{fixed}, so every link will have the
same \ac{MRAI} value.
The results are presented in \Cref{fig:clique_evolution}

\begin{figure}[h]
     \centering
     \begin{subfigure}[b]{0.45\textwidth}
         \centering
         \includegraphics[width=\textwidth]{images/clique/messagesVStime/pareto-clique-constant_mrai_evolution.pdf}
		 \caption{Network performances \textbf{with} \textbf{\ac{IW}}}
         \label{fig:clique_evolution_IW}
     \end{subfigure}
     \hfill
     \begin{subfigure}[b]{0.45\textwidth}
         \centering
         \includegraphics[width=\textwidth]{images/clique/messagesVStime/pareto-clique-noIW-constant_mrai_evolution.pdf}
		 \caption{Network performances \textbf{without} \textbf{\ac{IW}}}
         \label{fig:clique_evolution_noIW}
     \end{subfigure}
		\caption{Evolution of the network performances on the clique graph of \num{15}
			nodes using a fixed \ac{MRAI} from \num{0} to \num{60} seconds. \fxfatal{
			use the same interval in the y-axis?}}
        \label{fig:clique_evolution}
\end{figure}

Is possible to notice in \Cref{fig:clique_evolution} both the effect of \ac{MRAI}
and \ac{IW}.
Those plots represent the network performances in terms of convergence time and
number of messages transmitted to reach the convergence after the transmission 
of the signal \q{AW}.
The convergence time is represented by the average time from all the nodes in the 
network.
Each point in the plots is the average of the \num{10} runs with the \textit{fixed}
\ac{MRAI} value on the $x$ axis.
The left $y$ axis should be used with the convergence time, the green line, while
the second $y$ axis represent the number of messages transmitted, the red line.

The effects of the first one are present in both the plots but in two different
moments.
In \Cref{fig:clique_evolution_IW} \ac{MRAI} affects both the convergence time and
the number of messages around \SI{20}{\second} up to \SI{30}{\second}.
After the threshold of \SI{30}{\second} the effects of \ac{MRAI} are counterproductive,
the convergence time is negatively affected because the nodes starts to wait more
time without obtaining more useful information.
This can be see also in the number of messages that reaches a constant value.

In \Cref{fig:clique_evolution_noIW} we can see the same effect but with a higher
\ac{MRAI} value.
The number of transmitted messages reaches the constant value with an \ac{MRAI}
value around \SI{30}{\second}.
The effects of \ac{IW} can be saw also in the number of messages and the convergence
time with a low \ac{MRAI}, is possible to reach even \num{12000} messages while
with \ac{IW} the maximum value is around \num{6500} messages.


\section{Internet like graph}
\label{sec:bgp_mrai_internet_like}

The internet like environment is more complex than the clique one, but it permits
to have a more close vision of what can really happen on the Internet.
During my studies I used different topologies with \num{1000} nodes resembling 
the Elmokashfi properties \cite{elmokashfi2010scalability} already 
described in \Cref{subsec:internet_like_env}.

Using this graph I will look for possible correlation between \ac{MRAI} and other
factor that can influence the network.
First of all \ac{MRAI} has a dependence on how it is set, I'm going to compare
different \ac{MRAI} strategies that can be used on an Internet like graph.
Another influence factor could be the signal used as input, or even the position 
of the node that provoke the change.

\section{Strategy dependence}
\label{sec:bgp_mrai_strategy_dependance}

Like I mentioned before, the network performances depends on the \ac{MRAI} strategies
chosen.
For this reason the first point of my study is to point out this differences.
In order to do that, the first study that I would like to present is the one
that study how the standard protocol evolves on an Internet environment.

The property of the environment chosen are described in \Cref{tbl:internet_like_properties}

\begin{table}[h]
	\begin{center}
	\begin{tabular}{ || m{4cm}| m{8cm} || } 
	\hline
	Property & Value \\ 
	\hline \hline
	Seeds & $[1, 10]$ \\ 
	\hline
	Signaling & \q{AW} \\
	\hline
		Withdraws delay & Uniform distribution between \SI{1}{\second} and \SI{60}{\second} \\ 
	\hline
	Implicit withdraw & Active \\ 
	\hline
		MRAI & $[0, 60]$ \\
	\hline
	Link delay & Uniform distribution between \SI{0.012}{\second} and \SI{3}{\second} \\
	\hline
	\end{tabular}
\end{center}

	\caption{Internet like environment properties}
	\label{tbl:internet_like_properties}
\end{table}

The graph is an \textit{Internet like} graph with \num{1000} nodes.
The node that will execute the signal has been chosen randomly between 
all the nodes of type \q{C}.
This graph will be the same for all the experiments in this section.

For each \ac{MRAI} strategy, that I'm going to present, has been executed \num{61}
experiments, one for each possible value of \ac{MRAI}, for each experiments
thanks to the environment variable has been executed \num{10} runs.
In total for each \ac{MRAI} strategy has been run \num{610} different runs

As \ac{MRAI} strategies I decided to use the following two:
\begin{itemize}
	\item \textbf{\textit{Fixed}}; Every link will have the same
		\ac{MRAI} value;
	\item \textbf{\textit{DPC}}; This strategy assign a different 
		\ac{MRAI} value to each link depending on the centrality of the node \cite{milani2020improving}
\end{itemize}

The centrality metric used is called \ac{DPC} and thanks to the fact that has been
already demonstrated that is possible to calculate it in a distributed way \cite{milani2019BGP} I will
assume that it is calculated in advance and that every node knows it's own centrality to
set the timers.

To permit a comparison between those two different strategies a constraint on the
\ac{MRAI} assignment has been introduced, the $mean$ of all the timers in the network
must be equal between the two strategies.
For the \textit{Fixed} strategy this is constraint intrinsically respected.
For the \textit{DPC} strategy the timers are multiplied by a factor $k$ that
permits to keep the average equal.

The results of the first strategy are showed in \Cref{fig:internet_like_1000_constant_evolution}.

\begin{figure}[h]
     \centering
     \begin{subfigure}[b]{0.45\textwidth}
         \centering
         \includegraphics[width=\textwidth]{images/internet_like/1000/constantMRAI/internet_like-constant_mrai_evolution.pdf}
		 \caption{Network performances, messages VS convergence time with different
			\ac{MRAI} values}
         \label{fig:internet_like_1000_constant_evolution_evolution}
     \end{subfigure}
     \hfill
     \begin{subfigure}[b]{0.45\textwidth}
         \centering
         \includegraphics[width=\textwidth]{images/internet_like/1000/constantMRAI/internet_like-constant.pdf}
		 \caption{Pareto front of Messages VS Convergence time}
         \label{fig:internet_like_1000_constant_evolution_paretoFront}
     \end{subfigure}
		\caption{Evolution of the network performances on the \textbf{Internet Like} graph 
			of \num{1000} nodes using a fixed \ac{MRAI} from \num{0} to \num{60} seconds.}
        \label{fig:internet_like_1000_constant_evolution}
\end{figure}

As is possible to see in \Cref{fig:internet_like_1000_constant_evolution_evolution}
without \ac{MRAI} we would have a low convergence time, dictated mostly by 
network delays and processing time. With, on the other hand an enormous amount
of messages.
Slightly increasing the \ac{MRAI} value, the number of messages will fell down
reaching a constant value around \num{8000}, while the convergence time
continuously grows linearly, as it happened for the clique graph in \Cref{fig:clique_evolution}.
This continuous linear grow is dictated by the fact that nodes keep meaningful
information for more time before sharing them with their neighbourhood.
\Cref{fig:internet_like_1000_constant_evolution_paretoFront} represent the Pareto
front of those experiments.
The Pareto frontier is the set of values that are Pareto efficient, this concept
has been already used in engineering to define the set of best outcomes from
the trade-off of two different parameters \cite{goodarzi2014introduction}.
We can clearly see that the majority of the points is concentrated to the left
of the chart, this means that few \ac{MRAI} values would give as result
a high number of messages and a small convergence time.
While, multiple \ac{MRAI} values would concentrate around the same value of 
messages transmitted.
This can confirm the fact that \ac{MRAI} would not influence messages
after a certain threshold but only the convergence time.

The results of the same environment without \ac{IW} are showed in 
\Cref{fig:internet_like_1000_constant_evolution_noIW}.

\begin{figure}[h]
     \centering
     \begin{subfigure}[b]{0.45\textwidth}
         \centering
         \includegraphics[width=\textwidth]{images/internet_like/1000/constantMRAI/internet_like-constant-noIW_mrai_evolution.pdf}
		 \caption{Network performances, messages VS convergence time with different
			\ac{MRAI} values}
         \label{fig:internt_like_1000_constant_noIW_evolution_evolution}
     \end{subfigure}
     \hfill
     \begin{subfigure}[b]{0.45\textwidth}
         \centering
         \includegraphics[width=\textwidth]{images/internet_like/1000/constantMRAI/internet_like-constant-noIW.pdf}
		 \caption{Pareto front of Messages VS Convergence time}
         \label{fig:internt_like_1000_constant_noIW_evolution_paretoFront}
     \end{subfigure}
		\caption{Evolution of the network performances on the \textbf{Internet Like} graph 
			of \num{1000} nodes using a fixed \ac{MRAI} from \num{0} to \num{60} seconds.
			\textbf{Without \ac{IW}}}
        \label{fig:internet_like_1000_constant_evolution_noIW}
\end{figure}

Also in this case, comparing \Cref{fig:internet_like_1000_constant_evolution_noIW,fig:internet_like_1000_constant_evolution},
is possible to notice that \ac{IW}
helps to reduce the number of messages and the convergence time without impacting
the network performances trend.

The second strategy, the one dependant on the \ac{DPC}, produced the results
in \Cref{fig:internet_like_1000_dpc_evolution}
As mentioned before, all the timers are adjusted to respect the same mean as in 
the \textit{fixed} \ac{MRAI} experiments.
For this reason points with the same \ac{MRAI} value are comparable one another.

\begin{figure}[h]
     \centering
     \begin{subfigure}[b]{0.45\textwidth}
         \centering
         \includegraphics[width=\textwidth]{images/internet_like/1000/dpc/internet_like-DPC_mrai_evolution.pdf}
		 \caption{Network performances, messages VS convergence time with different
			\ac{MRAI} values}
         \label{fig:internt_like_1000_DPC_evolution_evolution}
     \end{subfigure}
     \hfill
     \begin{subfigure}[b]{0.45\textwidth}
         \centering
         \includegraphics[width=\textwidth]{images/internet_like/1000/dpc/internet_like-DPC.pdf}
		 \caption{Pareto front of Messages VS Convergence time}
         \label{fig:internt_like_1000_DPC_evolution_paretoFront}
     \end{subfigure}
		\caption{Evolution of the network performances on the \textbf{Internet Like} graph 
			of \num{1000} nodes using a \textit{DPC} \ac{MRAI} strategy
			with an $MRAI_{mean}$ from \num{0} to \num{60} seconds.}
        \label{fig:internet_like_1000_dpc_evolution}
\end{figure}

\fxfatal{Cumbersome, reading again it's not very clear what I'm explaining}
This second strategy leads to the performances showed in \Cref{fig:internet_like_1000_dpc_evolution},
is possible to notice that the number of messages transmitted fell down
very quickly and it reaches the convergence value around an \ac{MRAI} value of
\num{10}.
But, it is also noticeable that there are a lot more spikes in this trend, that
deviate more from the constant value around \num{9000} messages.
Also the convergence time is affected by this behaviour.

\fxfatal{Consider introducing a figure to show both trend in the same plot}
Comparing \Cref{fig:internet_like_1000_dpc_evolution,fig:internet_like_1000_constant_evolution}.

is possible to notice that the two strategies leads to a different trend.
Both are equal at the beginning with \ac{MRAI} equal \num{0} but after a while
both the number of transmitted messages and the convergence time diverge.
The number of messages with the \ac{DPC} strategy variate more and it converge
around \num{9000} messages, while the \textit{fixed} strategy reaches \num{8000}
messages.
And the convergence time with the second strategy grows more quickly.
This is caused by the central clique of tier one nodes that have a high \ac{MRAI}
value.
The high \ac{MRAI} value is caused by the fact that all the leafs has \num{0.0}
as centrality that cause an \ac{MRAI} value of \num{0} and to respect 
the $MRAI_{mean}$ value the central nodes needs a huge \ac{MRAI}.
For example, with an $MRAI_{mean}$ of \SI{30}{\second} the node \num{1} (that is
one of the central clique nodes) has an \ac{MRAI} value of \num{79.35} for all its
neighbours.


The standard value of \ac{MRAI} is \SI{30}{\second} as described in
\cite{rfc4271} so I compared those strategies performances in a box-plot in 
\Cref{fig:boxplot_internet_like_1000}.
I decided to run \num{100} different runs for each strategy with the $MRAI_{mean}$
fixed to \SI{30}{\second}.

\begin{figure}[h]
     \centering
     \begin{subfigure}[b]{0.45\textwidth}
         \centering
         \includegraphics[width=\textwidth]{images/internet_like/1000/comparison/comparison_messages_boxplot.pdf}
		 \caption{Network performances, messages necessary to reach convergence
			with different \ac{MRAI} strategies}
         \label{fig:boxplot_internet_like_1000_messages}
     \end{subfigure}
     \hfill
     \begin{subfigure}[b]{0.45\textwidth}
         \centering
         \includegraphics[width=\textwidth]{images/internet_like/1000/comparison/comparison_time_boxplot.pdf}
		 \caption{Network performances, time required to reach convergence
			with different \ac{MRAI} strategies}
         \label{fig:boxplot_internet_like_1000_time}
     \end{subfigure}
	 \caption{Network performances comparison with different \ac{MRAI} strategies,
		Graph internet like with \num{1000} nodes, \ac{MRAI} value 
		\SI{30}{\second}, number of runs for each strategy \num{100}}
        \label{fig:boxplot_internet_like_1000}
\end{figure}

In \Cref{fig:boxplot_internet_like_1000} we can compare those two strategies,
the first figure, \Cref{fig:boxplot_internet_like_1000_messages} represent 
the number of messages transmitted by the \num{100} runs, we can see that
the two strategies, without \ac{IW}, are really close one another.
While in the time required for convergence, \Cref{fig:boxplot_internet_like_1000_time}
there are some huge difference between the two strategies, is not negligible 
that with the \ac{DPC} strategy the time required is almost the double of the
standard time.

In conclusion we can say that the \ac{MRAI} strategy is one of the factor that 
can influence the Network performances. 

\fxfatal{Maybe I can introduce more strategies to expand this section}

\section{Pareto Efficiency Front}
\label{sec:bgp_mrai_pareto_front}

The strategies exposed in \Cref{sec:bgp_mrai_strategy_dependance} are just few
of the possibilities that are available.
For this reason I would like to explore the set of possibilities looking 
for \ac{MRAI} configuration randomly generated.

I would then study the space of possibilities that are generated through the 
Pareto efficiency plot and compare the results with the Pareto efficiency
graphs.
To permit this comparison I would set \ac{MRAI} randomly but like for
the \ac{DPC} strategy respecting the average required.

\fxfatal{Insert table with the value of \ac{MRAI} used and the ranges, and
		 the number of total experiments executed}

\fxfatal{Insert the resulting plot}

\section{Signal dependance}
\label{sec:bgp_mrai_signal_dependance}

I would like to analyze how much the signal can impact the convergence performances
with the two different strategies of \Cref{sec:bgp_mrai_strategy_dependance}.

For this reason I used the same environment described before and execute the
experiments with different input signals from the same node, \q{AWA}, \q{AWAW}
and \q{AWAWA}.

In those experiments plays a role also the \q{\textit{re-advertisement distribution}}
for the second and third \q{A}, it has been set to a uniform distribution
between \SI{1}{\second} and \SI{60}{\second}, like the \q{\textit{withdraw distribution}}.

For those experiments I didn't evaluate the case with \ac{IW} deactivated.
\fxfatal{Explain why}

\fxfatal{all the plots has an \ac{MRAI} steep of \SI{10}{\second} to give a 
hint on the trend, redo the plots with a step of \SI{1}{\second}}

In \Cref{fig:internt_like_1000_evolution_AWA} is possible to see the evolution
for the signal \q{AWA}.

\begin{figure}[h]
     \centering
     \begin{subfigure}[b]{0.45\textwidth}
         \centering
         \includegraphics[width=\textwidth]{images/internet_like/1000/signals/AWA/constant/internet_like-constant_AWA_mrai_evolution.pdf}
		 \caption{Network performances, \textit{fixed} \ac{MRAI} strategy}
         \label{fig:internet_like_1000_fixed_AWA}
     \end{subfigure}
     \hfill
     \begin{subfigure}[b]{0.45\textwidth}
         \centering
         \includegraphics[width=\textwidth]{images/internet_like/1000/signals/AWA/dpc/internet_like-DPC_AWA_mrai_evolution.pdf}
		 \caption{Network performances, \ac{DPC} \ac{MRAI} strategy}
         \label{fig:internet_like_1000_dpc_AWA}
     \end{subfigure}
	 \caption{Network performances comparison with different \ac{MRAI} strategies,
		Graph internet like with \num{1000} nodes, signal \q{AWA}}
        \label{fig:internt_like_1000_evolution_AWA}
\end{figure}

Is possible to notice in \Cref{fig:internt_like_1000_evolution_AWA} a huge difference
in respect of the plots in \Cref{fig:internet_like_1000_constant_evolution,fig:internet_like_1000_dpc_evolution}.
The \ac{DPC} strategy was able to outcome the standard \textit{fixed} strategy
over multiple prospective.
Analyzing \Cref{fig:internet_like_1000_dpc_AWA} is possible to notice that
the red curve, the one that refers to the number of messages transmitted has
a very fast fell, with an \ac{MRAI} timer \textit{mean} of \SI{30}{\second}
the number of messages is less than $1/4$ in respect of an \ac{MRAI} \textit{mean}
of \SI{0}{\second}.
The convergence time curve has a completely different trend in respect of the
previous experiments.
We can notice some steps trend.
This is caused by the fact that now the timer is able to effectively act on the signal.
\ac{MRAI} doesn't affect the first message, in this case the first \q{A} of the
signal, but it can affect the next two messages.
Infact, some nodes are able to cache both the \q{WA} part of the signal and
completely avoid to send anything at all, because them have already transmitted
the first \q{A}.
The complete compression of the signal \q{AWA} is \q{A}.
The other evolution, for the \q{AWAW} and \q{AWAWA} signals, are showed in 
\Cref{fig:internt_like_1000_evolution_AWAW,fig:internt_like_1000_evolution_AWAWA}

Like before, comparing the standard \SI{30}{\second} fixed \ac{MRAI} I executed
\num{100} different runs for each strategy and each different signal, the results
are exposed in \Cref{fig:boxplot_internet_like_1000_time_allSignals}.

\begin{figure}[h]
     \centering
     \begin{subfigure}[b]{0.45\textwidth}
         \centering
         \includegraphics[width=\textwidth]{images/internet_like/1000/comparison/comparison_allSignals_messages_boxplot.pdf}
		 \caption{Network performances, messages necessary to reach convergence
			with different \ac{MRAI} strategies}
         \label{fig:boxplot_internet_like_1000_messages_allSignals}
     \end{subfigure}
     \hfill
     \begin{subfigure}[b]{0.45\textwidth}
         \centering
         \includegraphics[width=\textwidth]{images/internet_like/1000/comparison/comparison_allSignals_time_boxplot.pdf}
		 \caption{Network performances, time required to reach convergence
			with different \ac{MRAI} strategies}
         \label{fig:boxplot_internet_like_1000_time_allSignals}
     \end{subfigure}
	 \caption{Network performances comparison with different \ac{MRAI} strategies,
		Graph internet like with \num{1000} nodes, \ac{MRAI} value 
		\SI{30}{\second}, number of runs for each strategy \num{100}, signal \q{allSignals}}
        \label{fig:boxplot_internet_like_1000_allSignals}
\end{figure}

Is possible to notice in \Cref{fig:boxplot_internet_like_1000_allSignals} that
both the strategies has different performances in respect of the signal 
produced by the single node source.
In particular performances are better when the signal ends up with an \q{A}.
That's because, after the first \q{A}, giving the \ac{MRAI} timer long enough, 
a node is able to compress a sequence that ends with an other \q{A} to the
empty set and don't send anything more.
While if the sequence ends up with an \q{W} it has to, at least, send an other
message to notify the withdraw.
%And withdraws are not affected by \ac{MRAI} like specified in a IETF draft of
%\num{2012} named \q{Revisions to the BGP 'Minimum Route Advertisement Interval'}.

Other than that is possible to notice that the \ac{DPC} techniques has better
results in terms of messages transmitted, while it could have an higher 
convergence time.
This is caused like before by the high \ac{MRAI} values used by the most
central nodes.

In conclusion there is a correlation between \ac{MRAI} and the sequence of messages
transmitted by the source node.
In particular more the timer is able to compress sequence more the performances
are good.

\fxfatal{consider moving figures from the appendix to the chapter}

\section{Position dependance}
\label{sec:position_dependance}

The last factor of influence for \ac{MRAI} that I would like to study is how much
the position of the signal source can influence the convergence.
The main hypothesis is that a node closer to the central clique, that generates
a signal would provoke a message storm bigger in respect of a node on the perimeter
of the network.
This is true only if \ac{MRAI} is large enough to block the storm near the source
of it exporting only the correct information at the end of it.

\subsection{Different signal sources}
\label{subsec:different_destinations}

As first try I have decided to try \num{10} different destination chosen randomly
on the same graph, this graph is an Internet like topology with \num{1000} nodes.
After that I run the same environment with all the different destination.
I also used different \ac{MRAI} strategies, repeating the experiments for all of
them.
With this results is possible to analyze how different signal sources provoke 
different network performances and also study how different \ac{MRAI} strategies
adapt with different nodes that provoke messages storms.

\subsection{Hierarchical influence}
\label{subsec:hierarchical_influence}

What about the position in the hierarchy?
Internet is very strong hierarchical graph, \Cref{fig:internet_graph_hierarchical}
is an example with a small set of nodes but its possible to define different levels
of the graph.
If we take the central clique as the rout of the graph then all the nodes will
be at a certain distance (in terms of hops) from it.

Nodes that are on the same hierarchical level reacts on the same way? 

\begin{itemize}
    \item And how much is influencing the position?
    \item Hierarchically?
\end{itemize}
