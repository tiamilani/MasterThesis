%%%%%%%%%%%%%%%%%%%%%%%%%%%%%%%%%%%%%%%%%%%%%%%
%
% Template per Elaborato di Laurea
% DISI - Dipartimento di Ingegneria e Scienza dell’Informazione
%
% update 2015-09-10
%
% Per la generazione corretta del 
% pdflatex nome_file.tex
% bibtex nome_file.aux
% pdflatex nome_file.tex
% pdflatex nome_file.tex
%
%%%%%%%%%%%%%%%%%%%%%%%%%%%%%%%%%%%%%%%%%%%%%%%

% formato FRONTE RETRO
\documentclass[epsfig,a4paper,11pt,titlepage,twoside,openany]{book}
\usepackage{epsfig}
\usepackage{plain}
\usepackage{setspace}
\usepackage[paperheight=29.7cm,paperwidth=21cm,outer=1.5cm,inner=2.5cm,top=2cm,bottom=2cm]{geometry} % per definizione layout
\usepackage{titlesec} % per formato custom dei titoli dei capitoli

%%%%%%%%%%%%%%
% supporto lettere accentate
%
%\usepackage[latin1]{inputenc} % per Windows;
\usepackage[utf8x]{inputenc} % per Linux (richiede il pacchetto unicode);
%\usepackage[applemac]{inputenc} % per Mac.

\singlespacing

\usepackage[english]{babel}

\begin{document}

  % nessuna numerazione
  \pagenumbering{gobble} 
  \pagestyle{plain}

\thispagestyle{empty}

\begin{center}
  \begin{figure}[h!]
    \centerline{\psfig{file=images/marchio_unitrento_colore_it_202002.eps,width=0.6\textwidth}}
  \end{figure}

  \vspace{2 cm} 

  \LARGE{Dept. of Information Engineering and Computer Science\\}

  \vspace{1 cm} 
  \Large{Graduate degree in\\
	Computer Science
  }

  \vspace{2 cm} 
  \Large\textsc{Final elaborate\\} 
  \vspace{1 cm} 
  \Huge\textsc{Title\\}
  \Large{\it{Subtitle (optionl)}}


  \vspace{2 cm} 
  \begin{tabular*}{\textwidth}{ c @{\extracolsep{\fill}} c }
  \Large{Supervisors} & \Large{graduating student}\\
  \Large{......}& \Large{Milani Mattia}\\
  \end{tabular*}

  \vspace{2 cm} 

  \Large{Accademic Year 2019/2020}
  
\end{center}



  \clearpage
 
  \thispagestyle{empty}

\begin{center}
  {\bf \Huge Thanks}
\end{center}

\vspace{4cm}


\emph{
  ...thanks to...
}

  \clearpage
  \pagestyle{plain} % nessuna intestazione e pie pagina con numero al centro

  
    % inizio numerazione pagine in numeri arabi
    \mainmatter

    % indice
    \tableofcontents
    \clearpage
    
    % gruppo per definizone di successione capitoli senza interruzione di pagina
    \begingroup
      % redefinizione del formato del titolo del capitolo
      % da formato
      %   Capitolo X
      %   Titolo capitolo
      % a formato
      %   X   Titolo capitolo
      
      \titleformat{\chapter}
        {\normalfont\Huge\bfseries}{\thechapter}{1em}{}
        
      \titlespacing*{\chapter}{0pt}{0.59in}{0.02in}
      \titlespacing*{\section}{0pt}{0.20in}{0.02in}
      \titlespacing*{\subsection}{0pt}{0.10in}{0.02in}
      
      % sommario
      \input{sommario}
      
      %%%%%%%%%%%%%%%%%%%%%%%%%%%%%%%%
      % lista dei capitoli
      %
      % \input oppure \include
      %
      \input{capitolo1}
      \input{capitolo2}
      \input{capitolo3}
      
    \endgroup


    % bibliografia in formato bibtex
    %
    % aggiunta del capitolo nell'indice
    \addcontentsline{toc}{chapter}{Bibliografia}
    % stile con ordinamento alfabetico in funzione degli autori
    \bibliographystyle{plain}
    \bibliography{biblio}

    \titleformat{\chapter}
        {\normalfont\Huge\bfseries}{Appendix \thechapter}{1em}{}
    % sezione Allegati - opzionale
    \appendix
    \input{allegati}

\end{document}
