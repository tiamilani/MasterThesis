%%%%%%%%%%%%%%%%%%%%%%%%%%%%%%%%%%%%%%%%%%%%%%%
%
% Template per Elaborato di Laurea
% DISI - Dipartimento di Ingegneria e Scienza dell’Informazione
%
% update 2015-09-10
%
% Per la generazione corretta del 
% pdflatex nome_file.tex
% bibtex nome_file.aux
% pdflatex nome_file.tex
% pdflatex nome_file.tex
%
%%%%%%%%%%%%%%%%%%%%%%%%%%%%%%%%%%%%%%%%%%%%%%%

% formato FRONTE RETRO
\documentclass[epsfig,a4paper,11pt,titlepage,twoside,openany]{book}
\usepackage{epsfig}
\usepackage{plain}
\usepackage{setspace}
\usepackage[paperheight=29.7cm,paperwidth=21cm,outer=1.5cm,inner=2.5cm,top=2cm,bottom=2cm]{geometry} % per definizione layout
\usepackage{titlesec} % per formato custom dei titoli dei capitoli

\usepackage[utf8x]{inputenc} 

\singlespacing

\usepackage[british]{babel}

%% My packages
\usepackage{silence}\WarningsOff[latexfont]
\usepackage{amsmath}
\usepackage{amsfonts}
\usepackage{amssymb}
\usepackage{graphicx}
\graphicspath{images/}
\usepackage{cite}
\usepackage{url}
\usepackage{subcaption}
\usepackage{float}
\usepackage[ruled,vlined,linesnumbered]{algorithm2e}
\SetKwProg{Fn}{Event}{}{}
\SetKw{And}{and}
\usepackage[binary-units,per-mode=symbol]{siunitx}
\sisetup{list-final-separator = {, and },detect-weight=true, detect-family=true}
\usepackage{booktabs}
\usepackage{pifont}
\usepackage{microtype}
\usepackage{textcomp}
\usepackage[capitalise]{cleveref}
\def\figname{\csname cref@figure@name\endcsname\xspace}
\def\tabname{\csname cref@table@name\endcsname\xspace}
\def\secname{\csname cref@section@name\endcsname\xspace}
\def\eqpname{\csname cref@equation@name@plural\endcsname\xspace}
\crefname{algorithm}{Listing}{Lists.}
\Crefname{algorithm}{Listing}{Listings}
\SetAlgorithmName{Listing}{Listing}{List of Listings}
\crefname{lstlisting}{listing}{listings}
\Crefname{lstlisting}{Listing}{Listings}
\usepackage{xspace}
\usepackage{hyphenat}
\usepackage[draft,inline,nomargin,index]{fixme}
\fxsetup{theme=color}
\usepackage{grffile}
\usepackage{xfrac}
\usepackage{multirow}
\usepackage[font={small}]{caption}
\usepackage{imakeidx}

\usepackage{tikz}
\usetikzlibrary{calc,shapes,arrows,fit,positioning}

%listings configuration
\usepackage{listings}
\lstset{
   language=sh,
   columns=fixed,
   breaklines=true,
   breakatwhitespace=true,
   prebreak=\textbackslash,
   basicstyle=\ttfamily\small,
   showstringspaces=false,
   upquote=true,
   keywordstyle=\ttfamily\small
}

\usepackage{color}
\definecolor{gray}{rgb}{0.4,0.4,0.4}
\definecolor{darkblue}{rgb}{0.0,0.0,0.6}
\definecolor{cyan}{rgb}{0.0,0.6,0.6}

\lstdefinelanguage{XML}
{
  morestring=[b]",
  morestring=[s]{>}{<},
  morecomment=[s]{<?}{?>},
  stringstyle=\color{black},
  identifierstyle=\color{darkblue},
  keywordstyle=\color{cyan},
  morekeywords={xmlns,version,type}% list your attributes here
}

\definecolor{codegreen}{rgb}{0,0.6,0}
\definecolor{codegray}{rgb}{0.5,0.5,0.5}
\definecolor{codepurple}{rgb}{0.58,0,0.82}
\definecolor{backcolour}{rgb}{0.95,0.95,0.92}

\lstdefinestyle{shell}{
    backgroundcolor=\color{backcolour},
    breakatwhitespace=false,
    breaklines=true,
    captionpos=b,
    keepspaces=true,
    showspaces=false,
    showstringspaces=false,
    showtabs=false,
    tabsize=2
}

\lstdefinestyle{graphml}{
    backgroundcolor=\color{backcolour},
    breakatwhitespace=false,
    breaklines=true,
    captionpos=b,
    keepspaces=true,
    showspaces=false,
    showstringspaces=false,
    showtabs=false,
    tabsize=2
}

\lstMakeShortInline[language=bash]|

% fix cleveref and breqn
\makeatletter
\let\cref@old@eq@setnumberOld\eq@setnumber
\def\eq@setnumber{%
\cref@old@eq@setnumberOld%
\cref@constructprefix{equation}{\cref@result}%
\protected@xdef\cref@currentlabel{%
[equation][\arabic{equation}][\cref@result]\p@equation\eq@number}}
\makeatother

\RequirePackage{xstring}
\RequirePackage{xparse}
\RequirePackage[index=true]{acro}
\NewDocumentCommand\acrodef{mO{#1}mG{}}{\DeclareAcronym{#1}{short={#2}, long={#3}, #4}}
\NewDocumentCommand\acused{m}{\acuse{#1}}

% Acronim definition
\acrodef{ADV}{advertisement}
\acrodef{AS}{Autonomous System}{short-plural=es}
\acrodef{BGP}{Border Gateway Protocol}
\acrodef{BIRD}{BGP Internet Routing Daemon}
\acrodef{DPC}{Destination Partial Centrality}
\acrodef{eBGP}{Exterior BGP}
\acrodef{ERP}{Exterior Routing Protocol}
\acrodef{IP}{Internet Protocol}
\acrodef{MRAI}{Minimum Route Advertisement Interval}
\acrodef{NH}{Next Hop}
\acrodef{RFC}{Request For Comment} 
\acrodef{TCP}{Transmission Control Protocol}
\acrodef{FSM}{Finite State Machine}
\acrodef{DES}{Descrete Event Simulator}
\acrodef{RFD}{Route Flap Damping}
\acrodef{RNG}{Random Number Generator}

\newcommand{\figwidthfour}{0.78}
\newcommand{\figwidth}{0.78}
\newcommand{\figvspace}{-1.5em}

\begin{document}

  % nessuna numerazione
  \pagenumbering{gobble} 
  \pagestyle{plain}

\thispagestyle{empty}

\begin{center}
  \begin{figure}[h!]
    \centerline{\psfig{file=images/marchio_unitrento_colore_it_202002.eps,width=0.6\textwidth}}
  \end{figure}

  \vspace{2 cm} 

  \LARGE{Dept. of Information Engineering and Computer Science\\}

  \vspace{1 cm} 
  \Large{Graduate degree in\\
	Computer Science
  }

  \vspace{2 cm} 
  \Large\textsc{Final elaborate\\} 
  \vspace{1 cm} 
  \Huge\textsc{Title\\}
  \Large{\it{Subtitle (optionl)}}


  \vspace{2 cm} 
  \begin{tabular*}{\textwidth}{ c @{\extracolsep{\fill}} c }
  \Large{Supervisors} & \Large{graduating student}\\
  \Large{......}& \Large{Milani Mattia}\\
  \end{tabular*}

  \vspace{2 cm} 

  \Large{Accademic Year 2019/2020}
  
\end{center}



  \clearpage
 
  \thispagestyle{empty}

\begin{center}
  {\bf \Huge Thanks}
\end{center}

\vspace{4cm}


\emph{
  ...thanks to...
}

  \clearpage
  \pagestyle{plain} % nessuna intestazione e pie pagina con numero al centro

  % inizio numerazione pagine in numeri arabi
  \mainmatter

  % indice
  \tableofcontents
  \clearpage
  
  % gruppo per definizone di successione capitoli senza interruzione di pagina
  \begingroup
    % redefinizione del formato del titolo del capitolo
    % da formato
    %   Capitolo X
    %   Titolo capitolo
    % a formato
    %   X   Titolo capitolo
    
    \titleformat{\chapter}
      {\normalfont\Huge\bfseries}{\thechapter}{1em}{}
      
    \titlespacing*{\chapter}{0pt}{0.59in}{0.02in}
    \titlespacing*{\section}{0pt}{0.20in}{0.02in}
    \titlespacing*{\subsection}{0pt}{0.10in}{0.02in}
    
    % sommario
    \input{sommario}
    
    %%%%%%%%%%%%%%%%%%%%%%%%%%%%%%%%
    % lista dei capitoli
    %
    % \input oppure \include
    %
    \input{capitolo1}
    \input{capitolo2}
    \input{capitolo3}
    
  \endgroup


  % bibliografia in formato bibtex
  %
  % aggiunta del capitolo nell'indice
  \addcontentsline{toc}{chapter}{References}
  % stile con ordinamento alfabetico in funzione degli autori
  \bibliographystyle{IEEEtran}
  \bibliography{references}

  \titleformat{\chapter}
      {\normalfont\Huge\bfseries}{Appendix \thechapter}{1em}{}
  % sezione Allegati - opzionale
  \appendix
  \chapter{Appendix}

\begin{figure}[h]
     \centering
     \begin{subfigure}[b]{0.45\textwidth}
         \centering
         \includegraphics[width=\textwidth]{images/internet_like/1000/signals/AWAW/constant/internet_like-constant_AWAW_mrai_evolution.pdf}
		 \caption{Network perforcances, \textit{fixed} \ac{MRAI} strategy}
         \label{fig:internet_like_1000_fixed_AWAW}
     \end{subfigure}
     \hfill
     \begin{subfigure}[b]{0.45\textwidth}
         \centering
         \includegraphics[width=\textwidth]{images/internet_like/1000/signals/AWAW/dpc/internet_like-DPC_AWAW_mrai_evolution.pdf}
		 \caption{Network perforcances, \ac{DPC} \ac{MRAI} strategy}
         \label{fig:internet_like_1000_dpc_AWAW}
     \end{subfigure}
	 \caption{Network perfomances comparison with different \ac{MRAI} strategies,
		Graph internet like with \num{1000} nodes, signal \q{AWAW}}
        \label{fig:internt_like_1000_evolution_AWAW}
\end{figure}

\begin{figure}[h]
     \centering
     \begin{subfigure}[b]{0.45\textwidth}
         \centering
         \includegraphics[width=\textwidth]{images/internet_like/1000/signals/AWAWA/constant/internet_like-constant_AWAWA_mrai_evolution.pdf}
		 \caption{Network perforcances, \textit{fixed} \ac{MRAI} strategy}
         \label{fig:internet_like_1000_fixed_AWAWA}
     \end{subfigure}
     \hfill
     \begin{subfigure}[b]{0.45\textwidth}
         \centering
         \includegraphics[width=\textwidth]{images/internet_like/1000/signals/AWAWA/dpc/internet_like-DPC_AWAWA_mrai_evolution.pdf}
		 \caption{Network perforcances, \ac{DPC} \ac{MRAI} strategy}
         \label{fig:internet_like_1000_dpc_AWAWA}
     \end{subfigure}
	 \caption{Network perfomances comparison with different \ac{MRAI} strategies,
		Graph internet like with \num{1000} nodes, signal \q{AWAWA}}
        \label{fig:internt_like_1000_evolution_AWAWA}
\end{figure}

\begin{figure}[h]
     \centering
     \begin{subfigure}[b]{0.45\textwidth}
         \centering
         \includegraphics[width=\textwidth]{images/internet_like/1000/comparison/comparison_AWA_messages_boxplot.pdf}
		 \caption{Network perforcances, messages necessary to reach convergence
			with different \ac{MRAI} strategies}
         \label{fig:boxplot_internet_like_1000_messages_AWA}
     \end{subfigure}
     \hfill
     \begin{subfigure}[b]{0.45\textwidth}
         \centering
         \includegraphics[width=\textwidth]{images/internet_like/1000/comparison/comparison_AWA_time_boxplot.pdf}
		 \caption{Network perforcances, time required to reach convergence
			with different \ac{MRAI} strategies}
         \label{fig:boxplot_internet_like_1000_time_AWA}
     \end{subfigure}
	 \caption{Network perfomances comparison with different \ac{MRAI} strategies,
		Graph internet like with \num{1000} nodes, \ac{MRAI} value 
		\SI{30}{\second}, number of runs for each strategy \num{100}, signal \q{AWA}}
        \label{fig:boxplot_internet_like_1000_AWA}
\end{figure}

\begin{figure}[h]
     \centering
     \begin{subfigure}[b]{0.45\textwidth}
         \centering
         \includegraphics[width=\textwidth]{images/internet_like/1000/comparison/comparison_AWAW_messages_boxplot.pdf}
		 \caption{Network perforcances, messages necessary to reach convergence
			with different \ac{MRAI} strategies}
         \label{fig:boxplot_internet_like_1000_messages_AWAW}
     \end{subfigure}
     \hfill
     \begin{subfigure}[b]{0.45\textwidth}
         \centering
         \includegraphics[width=\textwidth]{images/internet_like/1000/comparison/comparison_AWAW_time_boxplot.pdf}
		 \caption{Network perforcances, time required to reach convergence
			with different \ac{MRAI} strategies}
         \label{fig:boxplot_internet_like_1000_time_AWAW}
     \end{subfigure}
	 \caption{Network perfomances comparison with different \ac{MRAI} strategies,
		Graph internet like with \num{1000} nodes, \ac{MRAI} value 
		\SI{30}{\second}, number of runs for each strategy \num{100}, signal \q{AWAW}}
        \label{fig:boxplot_internet_like_1000_AWAW}
\end{figure}

\begin{figure}[h]
     \centering
     \begin{subfigure}[b]{0.45\textwidth}
         \centering
         \includegraphics[width=\textwidth]{images/internet_like/1000/comparison/comparison_AWAWA_messages_boxplot.pdf}
		 \caption{Network perforcances, messages necessary to reach convergence
			with different \ac{MRAI} strategies}
         \label{fig:boxplot_internet_like_1000_messages_AWAWA}
     \end{subfigure}
     \hfill
     \begin{subfigure}[b]{0.45\textwidth}
         \centering
         \includegraphics[width=\textwidth]{images/internet_like/1000/comparison/comparison_AWAWA_time_boxplot.pdf}
		 \caption{Network perforcances, time required to reach convergence
			with different \ac{MRAI} strategies}
         \label{fig:boxplot_internet_like_1000_time_AWAWA}
     \end{subfigure}
	 \caption{Network perfomances comparison with different \ac{MRAI} strategies,
		Graph internet like with \num{1000} nodes, \ac{MRAI} value 
		\SI{30}{\second}, number of runs for each strategy \num{100}, signal \q{AWAWA}}
        \label{fig:boxplot_internet_like_1000_AWAWA}
\end{figure}

\clearpage


\end{document}
